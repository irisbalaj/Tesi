\pagestyle{fancy}
\fancyhf{}
\fancyhead{}
\fancyhead[RO, LE] {\thepage}
\fancyfoot{}

\section{Introduzione}
Lo stage, svoltosi presso l'azienda \textbf{Industries S.P.A.} e con la supervisione del tutor Fabrizio Pittalis, consisteva nello studio di fattibilità dell'aggiornamento del progetto CRM, per capire se reingegnerizzare il progetto in maniera più efficace.\\
Oltre allo studio di fattibilità, era richiesto lo sviluppo di una versione dimostrativa del progetto basata su un set ristretto di dati, che potesse mostrare l'utilità delle tecnologie selezionate dallo stagista, comprensiva di una struttura di reportistica di log da sviluppare con software utilizzati dall'azienda.\\
Implicito nello sviluppo di tale demo, era richiesto lo studio di tecnologie proprie di Industries S.P.A, tipiche di un'azienda tessile di larga scala.

\subsection{L'Azienda}
L'azienda è una sede amministrativa di Moncler, marchio italiano di alta moda specializzato in vestiario invernale.\\
Nata francese nel 1952, Moncler diventa italiana nel 1992 e ad oggi vanta circa 3500 dipendenti ed oltre 200 punti vendita in tutto il mondo; è quotata nella borsa di Milano dal 2013, e nel 2017 ha superato 1,1 miliardi di euro di fatturato.\\
Le sedi amministrative principali sono situate in Italia a Milano, Trebaseleghe (Padova, sede dello stage) e Piacenza, e nel resto del mondo in Giappone e Stati Uniti.

\subsection{Obiettivi di stage}
L'obiettivo dello stage era di valutare la complessità di un possibile aggiornamento del progetto CRM con tecnologie più recenti.\\
Il progetto CRM consiste nell'alimentazione di un database con dati relativi a capi vendibili in una determinata campagna vendite a cadenza stagionale.\\
Il progetto era stato inizialmente sviluppato nel 2002 e consiste nell'estrazione dei dati dal database Oracle aziendale e popolamento di file di testo che poi vengono inviati ad un fornitore, il quale si occupa di caricare i dati contenuti in tali file nei database a cui fanno riferimento le campagne vendite. Queste campagne nello specifico sono degli showroom situati a Milano, New York e Tokyo e ricevono dati diversi a seconda delle politiche relative ad ogni stato.\\
La richiesta dei Service Manager era quella di ridurre la dispersività del programma, data dalla creazione di numerosi file, mantendendo ovviamente la consistenza e possibilmente aumentando la velicità dell'esecuzione.

\subsection{Principali problematiche}
La maggior parte delle aziende nel settore della moda utilizza un gestionale particolare, Stealth 3000, del quale non esiste documentazione ufficiale online dato che viene personalizzato per ogni azienda che ne abbia la licenza, e ciò implica che tutte le logiche dell'azienda per salvare ed estrarre i dati relativi ad ogni aspetto del settore della moda debbano essere spiegate da una persona dedicata, nel caso di questo stage dal tutor aziendale, per cui l'accesso ad informazioni non legate a tecnologie e programmazione, era necessariamente rallentato, nonché di non banale comprensione data l'enorme vastita di contesti.