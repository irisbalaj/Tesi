\section{Analisi e Pianificazione}
\subsection{Riunione}
Il primo passo verso la realizzazione del progetto è stato quello di discuterne in una riunione assieme al tutor interno e ai Service Managers interessati all'eventuale upgrade.\\
I motivi principali per cui si stesse discutendo un approccio diverso al trasferimento dei dati relativi al progetto CRM, erano legati alla eccessiva dispersività del progetto esistente, che crea molti file di testo e necessita dell'affidamento ad un fornitore che si occupi del caricamento dei dati nei database relativi ad ogni stato (Italia, USA e Giappone).\\
La natura del sistema esistente era tale da consistere di schedulazioni impostate a diverse fasce orarie da parte di aziende diverse. La prima schedulazione era relativa alla creazione dei file da parte di Industries, e la seconda era impostata dal fornitore a 6 ore di distanza, di comune accordo, per il caricamento effettivo dei dati. Sebbene l'esecuzione del programma di Industries che genera i file fosse molto breve, si era deciso di mantenere un discreto margine di tempo per intervenire in caso di errori prima del caricamento dei dati da parte del fornitore.\\
Nell'ottica di interrompere le relazioni con tale fornitore, si è deciso di considerare l'ipotesi di caricare i dati direttamente nei database a cui si riferiscono gli showroom, avendone nel tempo ottenuto l'accesso diretto. Tali database sono di proprietà di un fornitore che si occupa della creazione anche dei software installati nei cari punti vendita, principalmente tablet, grazie ai quali i vari clienti possono visualizzare il campionario ed eseguire gli evenutali ordini. Gli ordini dei clienti vengono poi trasferiti nello stesso modo dei caricamenti, sempre via file e tramite un fornitore, con un processo inverso rispetto a quello della popolazione dei database con i campionari, generando dei report interni ad Industries, tramite Oracle Reports di cui l'azienda possiede la licenza.

\subsection{Piano di Lavoro}
 Piano di lavoro

