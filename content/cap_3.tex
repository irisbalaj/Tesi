\section{Studio delle Tecnologie}
\subsection{Stealth 3000}
La maggior parte delle aziende tessili di larga scala in Italia utilizza un ERP specifico, Stealth 3000, sviluppato dall'azienda italiana Dedagroup.\\
In quanto ERP connette tutti gli applicativi utilizzati dall'azienda, oltre ad essere un gestionale disegnato specificatamente per il mercato della moda. \\
Si tratta di un'applet Java, accessibile solo internamente all'azienda da Internet Explorer (fino a v.) e Firefox (fino a v.)%Versioni supportate Stealth
[...] Panoramica Stealth e Studio della piattaforma.

\subsection{Oracle PL/SQL}
L'ambiente di sviluppo preferenziale di Industries è Oracle PL/SQL, ambiente e linguaggio di proprietà di Oracle che permette di scrivere in linguaggio di programmazione simil-C, ovvero procedurale ma fortemente orientato al collegamento con il database.
[...] Panoramica Oracle PL/SQL e studio della piattaforma

\subsection{Oracle Reports}
Uno degli strumenti utilizzati dall'azienda fortemente collegato al database Oracle, è Oracle Reports, che permette di disegnare dei report e fare in modo che i dati estratti derivino dal database di riferimento.\\
Possono essere utilizzate tutte le funzioni sviluppate in quel database da PL/SQL e ciò permette una diretta relazione tra gli strumenti.
[...] Panoramica Oracle Reports e studio della piattaforma

\subsection{Tecnologia per l'aggiornamento}
Le possibilità per l'aggiornamento del programma erano limitate già di partenza. Senza utilizzare software di terze parti, dei quali ovviamente sarebbe necessaria una licenza, la scelta ricadeva sulla produzione di un prototipo utilizzando i DbLink.\\
I DbLink sono uno strumento messo a disposizione da Oracle per la comunicazione fra due database distinti. Sono molto comodi in caso i due databse siano dello stesso tipo (es. Oracle - Oralce) ma sono necessarie delle misure particolari in caso i due sistemi siano di tipo diverso, come nel caso del progetto CRM, in cui il databse di origine è Oracle mentre quello di destinazione è Sql Server