\fancyhead[LO, RE] {Valutazione Retrospettiva}
\section{Valutazione Retrospettiva}
\subsection{Obiettivi raggiunti}
Il prodotto generato alla fine dello stage è stato tale da poter essere utilizzato con minime modifiche nella prossima campagna vendite, rispettando quindi quanto desiderato dalla proposta iniziale.
L'applicazione creata non è comunque particolarmente veloce data la natura lenta dei DbLink, ma nel complesso i Service Manager sono stati soddisfatti del risultato. Sono state inoltre create documentazioni esaustive a supporto del prototipo in caso venga effettivamente deciso di implementare la soluzione anche senza la mia presenza nell'azienda.\\

\subsection{Difficoltà incontrate}
La maggior difficoltà incontrata è stata relativa allo studio delle logiche di immagazzinamento dei dati nel database, nello specifico quelli relativi a soggetti ed oggetti, essendo informazioni specifiche del mondo della moda, che di conseguenza non vengono spiegate a livello accademico e ci sono minime informazioni di pubblico dominio a riguardo.\\
Per quanto riguarda le sfide tecnologiche, ambientarsi con l'ambiente di PL/SQL che non avevo mai visto è stata un esperienza di grosso impatto ma con il tempo si fa velocemente l'abitudine in particolare perché tutto ciò che viene offerto, con cui non avevo familiarità, si rivela essere particolarmente utile. 

\subsection{Bilancio Formativo}
Nel complesso l'esperienza è stata estremamente positiva a livello personale, dato l'ambiente di lavoro a livello umano e professionale. Ho avuto la possibilità di lavorare ad un progetto che potenzialmente ha un'utilità materiale per l'azienda e nel farlo ho potuto vedere il mondo che sta dietro ad un'azienda di prestigio mondiale.\\
A livello tecnologico le competenze sono molto verticali, essendo richieste quasi esclusivamente nel mercato dell'informatica applicata all'industria della moda, ma lo studio di ciò che viene offerto da PL/SQL, pur essendo un editor che permette di scrivere codice specifico al mondo dei database a differenza di linguaggi di programmazione ad oggetti, è certamente una competenza molto importante che sono soddisfatto di aver sperimentato.\\
Il supporto degli studi universitari è stato importante grazie ai corsi di Basi di Dati, che mi ha dato competenze tali da permettermi di ambientarmi velocemente ad un ambiente mai visto prima, ed al corso di Ingegneria del Software per aver trasmesso la mentalità necessaria ad approcciare un progetto di grosse dimensioni, che richiede pianificazione e creazione di documentazione.