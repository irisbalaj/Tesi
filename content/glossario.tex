\subsection*{A}
\textbf{applet}\\
In informatica, un programma di piccole dimensioni eseguiti all'interno di uno più ampio per svolgere una specifica funzione.

\subsection*{C}
\textbf{Compile-Time}\\
In informatica, tempo di compilazione, indica la fase di programma in cui uno strumento definito compilatore, verifica la struttura sintattica del codice e ne salva lo stato.

\subsection*{D}
\textbf{Database}\\
In informatica, archivio di dati strutturato in modo da razionalizzare la gestione e l'aggiornamento delle informazioni e da permettere lo svolgimento di ricerche complesse.

\textbf{Debug}\\
In informatica, attività di correzione di errori logici contenuti nel codice di un software.

\textbf{Demo}\\
Versione dimostrativa di un prodotto, con contenuti limitati allo scopo di presentare le caratteristiche principali del prodotto.

\subsection*{E}
\textbf{ERP}\\
Enterprise Resource Planning (pianificazione delle risorse di un'azienda) è un software di gestione che integra tutti i processi di buisness rilevati di un'azienda (vendite, acquisti, gestione magazzino, produzione, ecc...).

\subsection*{I}
\textbf{IDE}\\
Un ambiente di sviluppo integrato (in lingua inglese Integrated Development Environment) in informatuca, è un software che aiuta i programmatori in fase di sviluppo del codice sorgente del programma.

\subsection*{L}
\textbf{Log}\\
In informatica, solitamente un file sequenzale aperto in scrittura reso disponibile per funzionalità di amministrazione e monitoraggio.
\subsection*{P}
\textbf{Package}\\
In informatica, un insieme di funzionalità raccolte in un unico contesto (pacchetto).

\subsection*{Q}
\textbf{Query}\\
In informatica, interrogazione ad un database per estrarre dati che soddisfino determinati criteri di ricerca.

\subsection*{R}
\textbf{Record}\\
In informatica, una riga di un set di righe estratte da un'interrogazione ad un database.

\textbf{Run-Time}\\
In informatica, tempo di esecuzione, ovvero il momento del ciclo di vita di un programma in cui viene eseguito.

\textbf{Reverse Engeneering}\\
Processo per il quale un prodotto creato artificialmente viene decostruito per comprenderne l'architettura o per estrarne informazioni.

\subsection*{S}
\textbf{Service Manager}\\
Figura aziendale a capo della pianificazione, progettazione e amministrazione di un progetto, servizio o richiesta da parte di un cliente o dell'azienda stessa.

\textbf{Software}\\
In informatica, termine che può riferirsi ad istruzioni memorizzate su uno o più supporti informatici che può rappresentare uno o più programma o semplici dati di informazione.