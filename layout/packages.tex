\begin{filecontents*}{xmpl.bib}
@misc{Mon,
author    = "Moncler S.p.A.",
    howpublished = "\url{http://aiweb.techfak.uni-bielefeld.de/content/bworld-robot-control-software/}"
}
@misc{plsql,
author    = "Oracle PL/SQL Developer",
    howpublished = "\url{https://blogs.oracle.com/oraclemagazine/building-with-blocks}"
}
@misc{rep,
author    = "Oracle Reports Builder",
    howpublished = "\url{https://docs.oracle.com/middleware/12213/formsandreports/build-reports/orbr\_concepts1001.htm\#RSBDR111}"
}
@misc{dblink,
author    = "Oracle Database Links",
    howpublished = "\url{https://docs.oracle.com/cd/B28359\_01/server.111/b28310/ds\_concepts002.htm\#ADMIN12083}"
}
\end{filecontents*}

\documentclass[twoside,12pt]{report}

\usepackage[italian]{babel}
\usepackage[T1]{fontenc} 
\usepackage[utf8]{inputenc}

\usepackage[a4paper]{geometry}
\usepackage{lipsum} 
\usepackage{textcomp} % package che permette di inserire caratteri speciali


\usepackage{fancyhdr} % package per header e footer

\usepackage{graphicx} % package per uso di immagini
\usepackage{subfig}

\usepackage{titlesec} % package che permette di usare il comando "paragraph" come subsubsubsection!

\usepackage{multirow} % package per multirow per tabelle
\usepackage[table]{xcolor} % package per avere sfondo su celle
%\usepackage{slashbox}%diagonale tabelle

\usepackage{graphbox}

\usepackage{booktabs} %lista item nelle tabelle

\usepackage{parskip} % imposta lo stile italiano per i paragrafi

\usepackage{caption} % package che modifica i caption di immagini, tabelle etc.

\usepackage{eurosym} % package che importa il simbolo dell'euro

\usepackage{float} % \begin{figure}[H] ... \end{figure}

\usepackage{longtable} % package che permette di avere tabelle su più pagine

\usepackage{changepage} % package che permette di aggiustare i margini e centrare tabelle e figure

\usepackage{listings} % Per inserire del codice sorgente formattato

%\usepackage{xparse}
%\usepackage{slashbox}%diagonale tabelle

\usepackage{tabularx} %% Load packages that you use
\usepackage{booktabs}
\usepackage{longtable}
\usepackage[toc,page]{appendix} %Appendice
\usepackage{lastpage}
\usepackage{listings}
\usepackage{comment}

\usepackage{natbib}
\usepackage[colorlinks=true,linkcolor  = black]{hyperref}% package che permette insert di url e collegamenti