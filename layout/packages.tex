\documentclass[12pt]{article}

\usepackage[italian]{babel}
\usepackage[T1]{fontenc} 
\usepackage[utf8]{inputenc}

\usepackage[a4paper]{geometry}
\usepackage{lipsum} 
\usepackage{textcomp} % package che permette di inserire caratteri speciali
\usepackage[colorlinks=true,linkcolor  = black]{hyperref}% package che permette insert di url e collegamenti

\usepackage{fancyhdr} % package per header e footer

\usepackage{graphicx} % package per uso di immagini
\usepackage{subfig}

\usepackage{titlesec} % package che permette di usare il comando "paragraph" come subsubsubsection!

\usepackage{multirow} % package per multirow per tabelle
\usepackage[table]{xcolor} % package per avere sfondo su celle
%\usepackage{slashbox}%diagonale tabelle

\usepackage{graphbox}

\usepackage{booktabs} %lista item nelle tabelle

\usepackage{parskip} % imposta lo stile italiano per i paragrafi

\usepackage{caption} % package che modifica i caption di immagini, tabelle etc.

\usepackage{eurosym} % package che importa il simbolo dell'euro

\usepackage{float} % \begin{figure}[H] ... \end{figure}

\usepackage{longtable} % package che permette di avere tabelle su più pagine

\usepackage{changepage} % package che permette di aggiustare i margini e centrare tabelle e figure

\usepackage{listings} % Per inserire del codice sorgente formattato

%\usepackage{xparse}
%\usepackage{slashbox}%diagonale tabelle

\usepackage{tabularx} %% Load packages that you use
\usepackage{booktabs}
\usepackage{longtable}
\usepackage[toc,page]{appendix} %Appendice
\usepackage{lastpage}